\chapter{Introduction}
\label{cha:introduction}

“Wikimedia is a global movement whose mission is to bring free educational content to the world”. \footnote{\url{https://www.wikimedia.org/}}


Under this movement, many projects have risen, first among all Wikipedia, the most read encyclopedia in history and one of the top 15 most popular sites in the world.\footnote{\url{https://wikipedia.org/}} It is free, multilingual, and maintained by a community of volunteers through a model of open collaboration. The project carries no advertisements and is hosted by the Wikimedia Foundation, an American non-profit organization funded mainly through user donations.


Wikipedia is considered the most successful open and free project, not controlled by any company, where anybody can freely collaborate. Since its initial release in 2001, Wikipedia has continuously grown and is now counting 233 million pages in more than three hundred different languages. In a month, the website received roughly seven billion visit and about 9 million edits.


Opposed to private companies with a community of similar size, the Wikimedia Foundation makes its data public and easy accessible to anybody interested.\footnote{\url{https://dumps.wikimedia.org/}} Each month, a new dump of Wikipedia is created and published. This large amount of data can be extremely useful to data scientists willing to work on it. A community of millions can be studied and analyzed in a way never thought possible just a few years ago.


Wikipedia can continue to work only thanks to volunteers, who spend their time helping this project to grow. It is of vital importance to keep this community united and motivated, understand its actions and motivations and figure out why certain actions are taken and what is their effect on the community.

\section{Our project}
\label{sec:ourproject}

Our team worked on a series of related projects called “Community Health Metrics: Understanding Editor Drop-off” in collaboration with the Wikimedia Foundation, Eurecat, and the University of Trento.~\footnote{\url{https://meta.wikimedia.org/wiki/Grants:Project/Eurecat/Community_Health_Metrics:_Understanding_Editor_Drop-off}}


As stated in the project idea: “… we plan to carry out an extensive study of the editor’s lifecycle. Special attention will be devoted to underrepresented groups according to social dimensions such as gender or geographic provenance. We will extend state of the art metrics to analyze different language editions, combining a computational approach with qualitative inspection of the findings, involving expert editors from the communities for this task when necessary. This will increase our understanding of factors that are important for community health in Wikipedia, and it will result in explainable metrics that can be applied to signal early if a page or set of pages are undergoing detrimental dynamics”.


We currently lack the knowledge to understand and prevent drop-off for experienced editors. Our goal is to get a clearer picture of the phenomenon through the analysis of editors’ life cycles. We want to understand the dynamics and the factor associated with editor drop-off, increasing awareness about m.the community health. Some research on this topic has already been done by members of our team\cite{miquelwikipedia}.

We want to generate metrics and indicators about Wikipedia pages and groups of users characterized by gender, country, and native language. The results will be made available to the community through a dashboard.

\section{My contribution}
\label{sec:problem}

My focus was the emotional analysis of users and talk pages of Wikipedia. We tried to identify factors associated with editors leaving the project.

As in all human interactions, emotions play a crucial role in our decisions. Emotions can influence our actions, and actions can influence our emotions. Understanding this correlation can unveil recurring patterns in users' interaction to help identify problems in a community and find solutions.


I have analyzed words used by editors in a different context, tried to understand their emotional weight in a discussion and how it reflects on a user. Through all user interaction with the community, I have reconstructed its life cycle, in particular, which action led to which emotions and vice-versa. This topic was already covered in different studies, using different lexicon dictionaries and smaller datasets in 2012~\cite{laniado2012emotions} and 2014 \cite{iosub2014emotions}.


I also contributed to the categorization of users based on their gender and the development of the dashboard.

