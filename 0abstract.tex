\chapter*{Abstract} % senza numerazione
\label{Abstract}

\addcontentsline{toc}{chapter}{Abstract} % da aggiungere comunque all'indice

Wikipedia is a free, open-source encyclopedia. Its content is written by a community of volunteer editors. It has grown through the years, becoming the biggest encyclopedia ever made. Wikipedia is composed of articles, which collects the actual encyclopedia and talk pages, where editors can interact which each other.

This project is a part of a larger project that wants to understand when and why Wikipedia editors stop to edit. As all online community, there are several factors that lead users to join or leave this project. Our goal is to discover which particular events in a editor life-cycle leads to quit contribution to Wikipedia. This information can be used to better the experience for all users, and increase the size of the community.

My focus was to analyze emotions expressed by editors in Wikipedia talk pages posts. We tried to understand what emotions and sentiments users express and receive while interacting with each other. We used used "NRC Word-Emotion Association Lexicon" to map a set of words to the emotions and sentiments they express. The data I generated, joined to the results from other team members, will be used to analyze users' life cycles.

All users interactions in Wikipedia talk pages were collected in the WikiConv dataset, including modifications and deletions of posts. We used this data to count emotions expressed by different users and on different pages. In particular, we analyzed users for each month since the day a user joined Wikipedia. To avoid privacy violations we grouped users by gender and roles in Wikipedia, and no result on a specific user was ever published.

Defining a user group was an aspect of particular relevance in this research. We used two different approaches to define their gender, UserBoxes on users' pages and the gender specified by each user in the Wikipedia settings. The roles were retrieved from a dataset generated by another team member.


All the results we computed were loaded into a Postgres database and are publicly available through a GraphQL endpoint. A web application was also developed to show simple charts with different metrics.